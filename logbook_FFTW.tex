\documentclass[letterpaper,11pt]{article}
\usepackage{salvaspackage}
\graphicspath{{./Figs/}}

%%%%%%    	title	    %%%%%%
\title{Logbook:  FFTW3}

\author{Pedraza-Espitia S.\\ \\< \url{git.io/salvador} >}
\date{}
%%%%%%%%%%%%%%%%%%

\begin{document}
\maketitle
\spacing{1.1}

%
%\problema{1}{Nproblem}
%
\section{Instalación}
Siguiendo las instrucciones en \url{http://www.fftw.org/fftw3_doc/Installation-on-Unix.html#Installation-on-Unix}

Descargar la última versión de fftw3, descomprimir (extraer en $\sim$\verb|/softw|, en mi caso se creo el directorio \verb|fftw-3.3.6-pl2|), 

En la terminal:
\begin{lstlisting}
cd ~/softw/fftw-3.3.6-pl2; mkdir ../fftw
./configure --prefix=$HOME/softw/fftw
make
sudo make install
cd ..
mv fftw-3.3.6-pl2 ~/sources
\end{lstlisting}

\section{Programa en C}
\label{sec:progC}
Hay tres partes esenciales para remarcar, la primera es la lectura de los datos de un archivo de datos que debe contener una columna de datos reales (float), la segunda parte usa las funciones de la biblioteca \verb|fftw.3| para obtener la transformada de fourier y la tercera parte guarda los datos de la salida a un archivo que automáticamente nombrará \verb|salida.fftw|.
\lstinputlisting[language=c++]{processdata_fftw.cpp}

El programa se guarda en un archivo \verb|processdata_fftw.cpp| y se compila con\\
\begin{lstlisting}
g++ -o processdata_fftw processdata_fftw.cpp -lfftw3 -I/home/salva/softw/fftw-3.3.6-pl2/fftw/include -L/home/salva/softw/fftw-3.3.6-pl2/fftw/lib
\end{lstlisting}

Se ejecuta con
\begin{lstlisting}
./processdata_fftw input.dat outputName
\end{lstlisting}
\section{Output fftw}
\label{sec:output}
Para graficar en gnuplot uso
\begin{lstlisting}
gnuplot -p -e "plot 'data.dat'"
\end{lstlisting}
\coi{8}{plotlogistica}{Ec logística $c=2.x$ .}

A continuación se ejecuta el código presentado en la \autoref{sec:progC} y se obtiene la gráfica de la \autoref{fftwlogistica}
\coi{8}{fftwlogistica}{Después de aplicar FFTW a los datos que se graficaron en la \autoref{plotlogistica}, se obtienen las magnitudes de la salida de FFTW (números complejos).}
% section output (end)
\FloatBarrier

\section{Transformada de Fourier Discreta en Python}
\label{sec:TFP}

\begin{equation}
	X(F) = \int x(t) e^{ -j 2 \pi F t  } \mathrm d t
	\label{eq:continua}
\end{equation}

% https://www.youtube.com/watch?v=mkGsMWi_j4Q&t=47
\begin{equation}
	X_k = \sum_{n=0}^{N-1} x_n \cdot e^{-\frac{j2\pi k n}{N}}
	\label{eq:discreta}
\end{equation}

Un objetivo es programar la transformada de Fourier discreta en \textsc{Python} y obtener el mismo resultado que los algoritmos implementados en FFTW

\lstinputlisting[language=Python]{DFTfromScratch.py}
% section TFP (end)

%$\termina$
%\{ \}
\bibliographystyle{plainnat}
\bibliography{references}
\end{document}